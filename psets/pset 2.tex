\documentclass{pset}
\usepackage{colored-theorems}

\renewcommand{\hmwkTitle}{2nd problem set}
\renewcommand{\hmwkDueDate}{23/08/2025}
\renewcommand{\hmwkClass}{Pointset Topology}
\renewcommand{\hmwkSummary}{}
\renewcommand{\hmwkAuthorName}{AZZEDDINE Yacine}
\renewcommand{\backgroundimg}{background1.png}
\WarningFilter{DuplicateLabels}{Duplicate}

\begin{document}

\maketitle

\pagebreak

\subsection{Examples}
Before any systematic study of a mathematical object. It is important to have a wide variety of examples to play with and test out hypotheses. In the last homework we presented a few examples. In this homework we present a few more that're, in my opinion, more illustrative. 
\begin{enumerate}[label=(\alph*)]
    \item \emph{The metric topology}, as introduced in the last lecture, starting from a metric space $(X, d)$, you can define a topology $\mcl O$ on $X$ by saying $O\in \mcl O$ if for all $x\in O$ there exists an $r>0$ such that $x\in B(x, r)\subset O$. This provides an extremely rich and versatile source of examples. For instance, all geometric objects that you can think of can be thought of as subsets of euclidean spaces (e.g. the circle, the sphere, the cube, the torus) and all subsets of metric spaces can be thought of as metric spaces themselves (by restricting the distance function) and all metric spaces can be made into topological spaces. Thereby, we are ready to give a topological structure to all our usual geometric objects. (see the next section for more)
    \begin{exercise}
        show the the metric topology on a discrete metric space is the discrete topology.
    \end{exercise}
    \begin{exercise}
        show that there's no metric on $\bR$ that endows it with the trivial topology
    \end{exercise}
    \begin{exercise}
        Let $X=\{a, b, c\}$ and $\mcl O = \{\varnothing, \{a\}, \{b, c\}, X\}$. Prove that $\mcl O$ is generated by no metric on $X$.
    \end{exercise}
    \item Taking $X=\bR$, we can already define a few different topologies on $\bR$: The trivial and the discrete topology, the metric topology on $\bR$ (often called the \emph{usual topology on $\bR$}) and any of the topologies defined in hw 1. Here's one more: Let $\mcl O'$ consist of all open sets $O$ such that for all $x\in O$, there exists a half open interval $[a, b)$ such that $x\in[a, b)\subset O$. It can be shown that that any open set in the usual topology of $\bR$ is also in $\mcl O$. That is, $\mcl O_u\subset \mcl O'$ where $\mcl O_u$ denotes the usual topology on $\bR$. In general, given a set $X$ and two topologies $\mcl O$ and $\mcl O'$ on $X$ such that $\mcl O\subset \mcl O$, we say that $\mcl O'$ is \emph{finer} than $\mcl O$ or $\mcl O$ is \emph{coarser} than $\mcl O'$
    \begin{exercise}
        Show that $\mcl O'$ defined above is finer than $\mcl O_u$
    \end{exercise}
    \begin{exercise}
        Is the relation ``$\mcl O'$ is finer than $\mcl O$'' a total order? Why or why not?
    \end{exercise}
    \item The next example can be thought of as another generalization of the usual topology on $\bR$. This time, instead of leveraging the metric space structure, we depend on its total order: Suppose $(X, \leq)$ is a totally ordered set. The \emph{order topology} consists of all sets $U \subset X$ such that for every $x \in U$, there exist $a, b \in X$ with $a < x < b$ such that $(a, b) \subset U$. Where the interval is defined in the obvious way.
    
\end{enumerate}

\subsection{Topological Jargon}
Before we continue exploring continuity, we lay out here a few definitions and general constructions that will be useful throughout your study of topology. The idea of this section is to get you used to a few ubiquitous constructions. Do not worry if these seem like this is too much new information, you will get used to all of these new definitions through practice. For the remaining of this section, let $(X, \mcl O)$ be a topological space.
\begin{definition}[closed sets]
    a subset $A\subset X$ is said to be closed if $A^c$ is open.
\end{definition}
It is important to note that the naming scheme can be a bit misleading. Indeed, sets are not doors, they can be either open or closed or both open and closed or neither at the same time
\begin{exercise}[\skipforward]
    Find an example of an open set, a closed set, a set which is both closed and open and a set which is neither. (you can take $X=\bR$ with its usual topology)
\end{exercise}
\begin{exercise}
    prove the following properties of closed sets:
    \begin{enumerate}[label=\roman*.]
        \item $X$ and $\varnothing$ are closed.
        \item finite union of closed sets is closed.
        \item infinite intersections of of closed sets is closed.
    \end{enumerate}
\end{exercise}
Next, we will introduce two operators that will turn any arbitrary subset $A$ of $X$ into either a closed or open set: 
\begin{definition}
    \textbf{The closure of $\mathbf{A}$ in $\mathbf{X}$} is defined as
    \[
    \overline{A} = \bigcap \{\, B \subseteq X : B \supseteq A \ \text{and}\ B \ \text{is closed in}\ X \,\}.
    \]
    and \textbf{the interior of $\mathbf{A}$ in $\mathbf{X}$} is
    \[
    \Int A = \bigcup \{\, C \subseteq X : C \subseteq A \ \text{and}\ C \ \text{is open in}\ X \,\}.
    \]
\end{definition}
\begin{definition}
    we say $A\subset X$ is a neighborhood of $x\in X$ if there exists an open subset $O\in \mcl O$ such that
    \[x\in O\subset A\]
\end{definition}
You can think of a neighborhood as a set of points that are ``close'' to $x$. You might ask, well, technically, $X$ is a neighborhood of $x$ does that mean all points are close to $x$? But the point is that you can make neighborhoods to be as ``small'' as you want. In a manner which is  analogous to how you can take $\epsilon>0$ to be as small as you want in the definition of continuity or convergence. (ask me about this during office hours! I can expand on this a bit more.)
\begin{exercise}[\skipforward]
    Prove the following characterizations of closure:
    \begin{enumerate}[label=\roman*.]
        \item $\overline{A}$ is the smallest closed set containing $A$ (what does this mean formally?)
        \item $x\in \overline{A}$ if and only if every neighborhood of $x$ contains a point of $A$
    \end{enumerate}
\end{exercise}
\begin{exercise}[\skipforward]
    Prove the following characterizations of the interior:
    \begin{enumerate}[label=\roman*.]
        \item $\Int A$ is the largest open set contained in $A$.
        \item $x\in \Int A$ if there exists a neighborhood of $x$ which is entirely contained in $A$
    \end{enumerate}
\end{exercise}
\begin{exercise}
    prove the following are equivalent:
    \begin{itemize}
        \item $A$ is open in $X$
        \item $A=\Int A$
        \item Every point of $A$ has a neighborhood contained in $A$
    \end{itemize}
\end{exercise}
\begin{exercise}
    prove that $A$ is closed in $X$ if and only if $A=\overline A$
\end{exercise}
Next, we define the \emph{boundary} of some arbitrary set $A$. Roughly speaking, a boundary point is a point which ``touches'' both the ``inside'' and the ``outside'' of a set. First, we define what we mean by the ``outside'' of a set
\begin{definition}
    Let $A\subset X$ we say that $x\in X$ is a boundary point of $A$ if and only if for all neighborhoods $N$ of $x$, $N\cap A\neq\varnothing$ 
    and $N\cap A^c\neq\varnothing$. The set of all boundary points is denoted $\partial A$.
\end{definition}
\begin{exercise}[\skipforward]
    prove the following assertions:
    \begin{enumerate}[label=\roman*.]
        \item $\partial A = \overline A\setminus \Int A$
        \item $\partial A\cap \Int A=\varnothing$
        \item $\overline A = \partial A \cup A = \partial A \cup \Int A$
    \end{enumerate}
\end{exercise}
Note that if $A$ is open then $\Int A = A$. Thereby, \romannumeral 2. and \romannumeral 3. reinforces the intuition that an open set contains none of its boundary points and a closed set contains all of its boundary points.
\begin{exercise}
    Suppose $A$ is open. Prove that $\partial(\partial A) = \partial A$. Can we drop the assumption that $A$ is open?
\end{exercise}

{\tiny ref. Lee ITM, section 1 of chapter 2.}

\subsection{Convergence and Continuity}
Now we come to the single most important definition of all of topology. It is important to note that, in the context of general topology, the following is a \emph{definition} and not a theorem.
\begin{definition}[continuity]
    Let $(X, \mcl O)$ and $(Y, \mcl O')$ be topological spaces. A function $f\colon X\longrightarrow Y$ is said to be continuous if for every open 
    set $A\subset Y$, $f\inv(A)$ is an open subset of $X$
\end{definition}
\begin{exercise}
Show that the composition of continuous functions is continuous. That is, if $f\colon X\longrightarrow Y$ and 
$g\colon Y\longrightarrow Z$ then $g \circ f\colon X\longrightarrow Z$ is continuous.
\end{exercise}
\begin{exercise}
    $f\colon (X, \mcl O) \longrightarrow (Y, \mcl O')$ is continuous if and only if the preimage of every closed set is closed.
\end{exercise}
We have shown in the previous lecture that, when you endow metric spaces with the metric topology, the above definition is equivalent to our usual notion of continuity using $\delta$ and $\epsilon$. Another notion which is captured through topology is convergence. This should come as no surprise since if you take a convergent sequence in some topological space, say, the plane, and you twist and bend and deform the plane in a continuous manner, the sequence would remain convergent. In a more formal language, if $f\colon \bR^2\longrightarrow\bR^2$ is a continuous function and $x_n\to x$ in $\bR^2$ then $f(x_n)\to f(x)$. This is a strong clue that convergence also can be detected through the topological structure of the space alone. To do that, we look again at metric spaces for inspiration. If we can find a a criterion which makes use of the open sets alone then we can use that as our definition of convergence in general! If you recall from the prereq problem sheet, problem 3.2 comes very close to doing that. However, the astute reader might notice that the theorem remains true if we merely replace $B(x, \eps)$ to any \emph{neighborhood} around $x$.
\begin{definition}[convergence]
    let $(X, \mcl O)$ be a topological space. We say that a sequence $(x_n)_{n\in\bN}$ converges to $x$ if for all neighborhoods $N$ of $x$, $x_n\in N$ for all but finitely many $n$.
\end{definition}
This further reinforces the intuition that neighborhoods gives a notion of what it means to be ``close'' to a given point $x$. Topologically speaking, neighborhoods are completely equivalent to $\epsilon$ balls, it's just that neighborhoods, being a tad bit more general are easier to work with in theory and far more easily generalizable to all topological spaces.
\begin{exercise}[\skipforward]
    prove that in the discrete topology, any convergent sequence is eventually constant
\end{exercise}
\begin{exercise}[\skipforward]
    Suppose $X$ is a topological space, $A$ is a subset of $X$, and $(x_i)$ is a sequence of points in $A$ that converges to a point $x\in X$. Show that $x\in\overline A$
\end{exercise}
\begin{definition}[sequential continuity]
    We say that a function $f\colon (X, \mcl O) \longrightarrow (Y, \mcl O')$ is sequentially continuous if for any sequence $x_n$ in $X$ which converges to $x$, $f(x_n)$ converges to $f(x)$
\end{definition}


\begin{exercise}[\skipforward]
    Now suppose $(X, d_X)$ and $(Y, d_Y)$ are metric spaces. Prove that continuity is equivalent to sequential continuity. Is the same true for topological spaces in general?
\end{exercise}

\end{document}