\documentclass{pset}
\usepackage{colored-theorems}

\renewcommand{\hmwkTitle}{2nd problem set}
\renewcommand{\hmwkDueDate}{16/08/2025}
\renewcommand{\hmwkClass}{Pointset Topology}
\renewcommand{\hmwkSummary}{}
\renewcommand{\hmwkAuthorName}{AZZEDDINE Yacine}
\renewcommand{\backgroundimg}{background1.png}
\WarningFilter{DuplicateLabels}{Duplicate}

\begin{document}

\maketitle

\pagebreak

\subsection{Topological Jargon}
Before we continue exploring continuity, we lay out here a few definitions and general constructions that will be useful throughout your study of 
topology. The idea of this section is to get you used to such ubiquitous constructions. Do not worry if the . For the remaining of this section, let $(X, \mcl O)$ be a topological space.
\begin{definition}[closed sets]
    a subset $A\subset X$ is said to be closed if $A^c$ is open.
\end{definition}
It is important to note that the naming scheme can be a bit misleading. Indeed, sets are not doors, they can be either open or closed or both open 
and closed or neither at the same time
\begin{exercise}[\skipforward]
    Find an example of an open set, a closed set, a set which is both closed and open and a set which is neither. (you can take $X=\bR$)
\end{exercise}
\begin{exercise}
    prove the following properties of closed sets:
    \begin{enumerate}[label=\roman*.]
        \item $X$ and $\varnothing$ are closed.
        \item finite union of closed sets is closed.
        \item infinite intersections of of closed sets is closed.
    \end{enumerate}
\end{exercise}
Next, we will introduce two operators that will turn any arbitrary subset $A$ of $X$ into either a closed or open set: 
\begin{definition}
    \textbf{The closure of $A$ in $X$}
    \[
    \overline{A} = \bigcap \{\, B \subseteq X : B \supseteq A \ \text{and}\ B \ \text{is closed in}\ X \,\}.
    \]
    and \textbf{the interior of $A$ in $X$}
    \[
    \Int A = \bigcup \{\, C \subseteq X : C \subseteq A \ \text{and}\ C \ \text{is open in}\ X \,\}.
    \]
\end{definition}
\begin{definition}
    we say $A\subset X$ is a neighborhood of $x\in X$ if there exists an open subset $O\in \mcl O$ such that
    \[x\in O\subset A\]
\end{definition}
You can think of a neighborhood as a set of points that are ``close'' to $x$. You might ask, well, technically, $X$ is a neighborhood of $x$ does 
that mean all points are close to $x$? But the point is that you can make neighborhoods to be as ``small'' as you want. In a manner which is 
analogous to how you can take $\epsilon>0$ to be as small as you want in the definition of continuity or convergence. (ask me about this during 
office hours! I can expand on this a bit more.)
\begin{exercise}[\skipforward]
    Prove the following characterizations of closure:
    \begin{enumerate}[label=\roman*.]
        \item $\overline{A}$ is the smallest closed set containing $A$ (what does this mean formally?)
        \item $x\in \overline{A}$ if and only if every neighborhood of $x$ contains a point of $A$
    \end{enumerate}
\end{exercise}
\begin{exercise}[\skipforward]
    Prove the following characterizations of the interior:
    \begin{enumerate}[label=\roman*.]
        \item $\Int A$ is the largest open set contained in $A$.
        \item $x\in \Int A$ if there exists a neighborhood of $x$ which is entirely contained in $A$
    \end{enumerate}
\end{exercise}
\begin{exercise}
    prove the following are equivalent:
    \begin{itemize}
        \item $A$ is open in $X$
        \item $A=\Int A$
        \item Every point of $A$ has a neighborhood contained in $A$
    \end{itemize}
\end{exercise}
\begin{exercise}
    prove that $A$ is closed in $X$ if and only if $A=\overline A$
\end{exercise}
Next, we define the \emph{boundary} of some arbitrary set $A$. Roughly speaking, a boundary point is a point which ``touches'' both the ``inside'' and 
the ``outside'' of a set. First, we define what we mean by the ``outside'' of a set
\begin{definition}
    We say that a point $x\in X$ is in the \textbf{exterior of $\mathbf A$} if there exists a neighborhood $N$ of $x$ such that $A\cap N = \varnothing$ and 
    we write $x\in \Ext A$
\end{definition}
It is easy to see that $\Ext A\subset A^c$
\begin{exercise}
    prove that $\Ext A=X\setminus \overline A$
\end{exercise}

{\tiny ref. Lee ITM, section 1 of chapter 2.}

\subsection{Convergence and Continuity}
\begin{definition}[continuity]
    Let $(X, \tau_X)$ and $(Y, \tau_Y)$ be topological spaces. A function $f\colon X\longrightarrow Y$ is said to be continuous for every open 
    set $A\subset Y$, $f\inv(A)$ is an open subset of $X$
\end{definition}
Note that, in the context of general topology, the above is a \emph{definition} and not a theorem.
\begin{definition}[convergence]
    
\end{definition}
\begin{definition}[sequential continuity]
    
\end{definition}

\begin{exercise}
Show that the composition of continuous functions is continuous. That is, if $f\colon X\longrightarrow Y$ and 
$g\colon Y\longrightarrow Z$ then $g \circ f\colon X\longrightarrow Z$ is continuous.
\end{exercise}

\begin{exercise}
Now suppose $(X, d_X)$ and $(Y, d_Y)$ are metric spaces. Prove that continuity is equivalent to sequential continuity.
\end{exercise}

\end{document}